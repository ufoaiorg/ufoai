%-*-texinfo-*-
%This is part of the "UFO:Alien Invasion"-Reference Manuals Tex sources.
%Copyright (C) 2006 Eric Goller.
%See the file ufo-manual_EN.tex for copying conditions.

%---------------------------------------------------------------------------
%							Version history
%
%  date		/		changes																	/		comment						/ 		author
%	06.10.06	/	rev. 0.0.5 (alpha) released											/											/	Elbenfreund
%	08.10.06	/	added some missing linebreaks at datasheets			/											/	Elbenfreund
%	09.10.06	/	added missing ufopedia entries									/	still missing shootgun		/	Elbenfreund
%
%---------------------------------------------------------------------------


\section{Ufopedia}
In the following you will find a list of selected Ufopedia entries. As a big part of UFO:AIs gameplay is about research and learning about your extraterrestrial enemy we do not want to spoil your fun by giving away all the secrets for free here. Also we have limited ourselves to list things that may be helpfull to get started with the game and make reasonable decisions without starting a new game a dozen times.

\subsection{Skills}
\subsubsection{Basic skills}
\paragraph*{Power}
 reflects a soldier's physical strength. A high physical strength is especially important for soldiers who handle heavy weapons and armor, as well as soldiers who fight in close quarters. Power directly influences the damage a soldier can do in melee combat, and how well a soldier is able to handle a weapon's recoil. Recoil decreases accuracy, so a soldier using a weapon with a lot of recoil needs strength to keep it pointing in the right direction when firing.\\
Power also affects soldier's health points (HP) and the amount of equipment a soldier can carry before he becomes encumbered. An encumbered soldier suffers a time units (TU) penalty, as well as an accuracy penalty.
\paragraph*{Speed}
 represents how fast a soldier moves. The attribute affects mainly how much time units (TU) a soldier has. However, TU are, arguably, the most important survival characteristics of a soldier, so the speed attribute should not be underestimated. Morever the skill determines the initiative (who shoots first) when reaction fire is triggered.
\paragraph*{Accuracy}
 represents how good a soldier is at hitting a target. Accuracy is important for all soldiers who use ranged weapons, but especially so for snipers, assault weapon specialists and explosive weapon users.
\paragraph*{Mind}
 is a representation of the mental training of a soldier. The better this attribute, the less likely a soldier is to panic, and the better the soldier is at psionic warfare. Moreover, the use of utility miscellaneous equipment, as well as mines, depends on this skill. Soldiers with weak minds should not count on fast promotions.
\subsubsection{Weapon Proficiencies}
\paragraph*{Close combat}
 skill represents a soldier's proficiency with close-range weapons. A soldier with a high Close Combat skill is better at aiming a pistol at a fast moving enemy, and can also wield a blade better in combat, increasing damage. Fire modes affected by the Close Combat skill are always short range and with relatively low TU costs.\\
Examples: Combaat knife, 9mm Pistol
\paragraph*{Heavy weapons}
 skill affects how well a soldier is able to handle weapons that weigh a lot or weapons that produce much recoil. Recoil decreases accuracy tremendously, unless the soldiers is trained to counteract with the muscles of his arms. Heavy weapons include high stopping power firearms such as flame throwers and shotguns, as well as many medium and long range heavy support fire weapons. Fire modes governed by the Heavy Weapons skill always trade accuracy for damage, but differ in every other characteristics.\\
Examples: Riot Shotgun, Flamethrower
\paragraph*{Assault guns}
 skill reflects a soldier's proficiency with assault weapons. The assault skill is a combination of the ability to quickly identify friend from foe in tight combat situations, and to fire a rifle at the latter, not the former. Weapons that use the Assault Guns skill tend to be good all-round weapons, with decent damage, accuracy and rate of fire.\\
Example: Assault Rifle, SMG
\paragraph*{Sniper rifles}
 skill represents a soldier's skill to aim a weapon very accurately, provided the weapon is designed for that. Most weapons used with that skill are indeed sniper rifles, but there are exceptions --- long range, high accuracy fire mode of any weapon is bound to require Sniper Rifles skill.\\
Example: Sniper rifle
\paragraph*{Explosives}
 skill represents a soldier's ability to use grenades, and weapons with a splash damage effect (whether caused by a high-explosive ammunition or any other ammunition, even alien). A soldier with a big High-Explosives skill is better at landing the charge at the spot where he wants it: at the feet of the enemy or, if the spot is unreachable, still close enough to harm him. A soldier trained in this skill will also be more attuned to a launcher's recoil and more accustomed to its fire trajectory, increasing accuracy.\\
Example: Rocket launcher, Frag granade

\newpage

\subsection{Primary Weapons}
The alien attack on Mumbai made our situation painfully clear. Their technology is far more advanced than ours. The complete inability of Commonwealth troops to make a dent in the Mumbai offensive revealed critical weaknesses in current military training and equipment. They lost three battalions just bringing the aliens to a standstill without inflicting significant casualties. PHALANX has to overcome these odds, and to do that we need the very best human technology has to offer.

The Excalibur Program was created to find the most effective weapons on Earth by reviewing their manufacturing standards, durability, operational record, and their combat performance in the situations where we've managed to bring the aliens to battle.
%hinweis aus munition f?r infos ?

\subsubsection*{Assault Rifle}
Technical Specifications: AR-80 Assault Rifle\\
CLASSIFIED LEVEL YELLOW\\
PHALANX Extraterrestrial Response Unit\\
Technical Document, Delta Clearance\\
Filed: 20 March 2084\\
By: Cdr. Paul Navarre, R\and D: Engineering Division, PHALANX, Atlantic Operations Command\\
\paragraph*{Overview}
The French LeBlanc FAA-191 (Fusil Assaut Automatique) provides the most advantageous mix of range, penetration against alien armour, and magazine size in the assault rifle class. It is a bullpup design, meaning the magazine and action of the weapon are located behind the grip to reduce overall weapon length. The 191 fires a 30-round box magazine of 4.7mm caseless ammunition, tungsten-cored steel penetrators with impressive armour-piercing capability over short and medium ranges. The round does not perform well at long range, but then assault rifles aren't meant for long-range firefights.

Though the design is far from new, first prototyped in 2057, no other rifle has fully surpassed the 191. It is accurate and quick to reload. Its rugged construction makes it dependable in combat and unlikely to break or jam. It is resistant to heat, cold, dust and humidity all at once. Ammunition and replacement parts are widely available due to the design's maturity and relative popularity among Earth's armed forces. This is a rifle you can entrust your life to.

For PHALANX use, we have given this rifle the classification AR-80.
\paragraph*{Recommended Doctrine}
This should be our go-to weapon for medium-range engagements. It can lay down an impressive amount of fire and is deadly out to far longer ranges than any pistol or SMG.

However, no one should entertain the illusion that the assault rifle is the end-all be-all of our arsenal. A balance of weapons is required for us to be able to deal with different situations; while the assault rifle remains a good weapon at close range, it can be slow to manoeuvre in tight spaces and is eclipsed by both submachine guns and shotguns at these ranges, weapons which provide a far greater point-blank punch. Any sweep of an urban building should be led by close-range weapons, while rifle-equipped soldiers either secure the perimeter or form a second assault wave to support the others.

If our rifle-equipped troops are caught beyond effective weapon range, they should make an advance from cover to cover in order to effectively bring their weapons to bear. Snipers should provide covering fire for the advancing teams or attempt to take out the enemy at range.
\paragraph*{Addenda}
While the AR-80 is a fine rifle in all respects, it was not designed to fight an alien invasion force. We should develop our own purpose-built weapons as soon as it becomes feasible to do so.

\newpage

\subsubsection*{S-1 Sniper Rifle}
Technical Specifications: S-1 Sniper Rifle\\
CLASSIFIED LEVEL YELLOW\\
PHALANX Extraterrestrial Response Unit\\
Technical Document, Delta Clearance\\
Filed: 20 March 2084\\
By: Cdr. Paul Navarre, R\and D: Engineering Division, PHALANX, Atlantic Operations Command\\
\paragraph*{Overview}
Originally an anti-materiel rifle, the Canada-built Forrester LRWS (Long Range Weapon System) has since been adopted by many countries as their principal sniper rifle. It is one of the bare handful of sniper rifles developed after 2040 that do not feature a bullpup configuration ('bullpup' meaning the magazine and action of the weapon are located behind the grip to reduce overall weapon length). It fires the massive 20mm HMG (Heavy Machine Gun) cartridge, fed by 5-round magazines which can weigh as much as one kilogramme apiece. The piston-retarded floating breech is equipped with an intricate gas dispersal system which decreases felt recoil to the level of an ordinary hunting rifle. This allows quick repeated shots on semi-automatic without any loss of accuracy.

The greatest advantages of the LRWS over other modern sniper rifles are its incredibly short barrel and light weight, the only rifle in its size class that can fire the 20mm HMG round. This is made possible by a uniquely-reinforced breech and barrel made almost entirely of tungsten and titanium alloys, able to withstand the force of the round's super-high-velocity powder. Due to its short barrel, designed for tactical urban situations, the LRWS is only effective out to approximately 1 kilometre -- barely a third of the range of a standard anti-materiel rifle -- but we estimate that PHALANX should never face a situation where this might become a problem. Its reduced accuracy is amended by a highly-advanced scope that will calculate and display intelligent bullet trajectories wherever the rifle is pointing.

The integral 'smart' bipod features a pneumatic suspension system that keeps the barrel perfectly horizontal to allow accurate fire even on broken ground. The buttstock and grip automatically mould themselves to fit any shooter. Most importantly, this rifle has racked up more alien kills in Mumbai than any other weapon deployed in the fighting.

At half the weight of other sniper rifles and twice the manoeuvrability, the LWRS offers the power and flexibility that our agents require. This weapon will not disappoint.

For PHALANX use, we have given this rifle the classification S-1.
\paragraph*{Recommended Doctrine}
Soldiers equipped with the S-1should keep their distance, fire from cover, and try to use aimed shots whenever possible. This is not an automatic weapon; a missed shot wastes time, ammunition and possibly life.

All our snipers should carry at least one backup sidearm such as the P-12 or the CRC-8 SMG, or a combat knife at the very least. Should aliens threaten a sniper at close range, he should immediately draw his sidearm. Under no circumstances should he try to use an S-1 to fend off attackers at close range. The S-1 is too slow-firing to stop an advancing alien and will quickly deplete its magazine if this is attempted. If the soldier tries to shoot his magazine dry before drawing his sidearm, it will be too late.
\paragraph*{Addenda}
Along with high-explosive rockets and grenades, this is one of the few standard-issue human weapons that are fully effective against robotic aliens.

\newpage

\subsubsection*{Flamethrower}
CLASSIFIED LEVEL YELLOW\\
PHALANX Extraterrestrial Response Unit\\
Technical Document, Delta Clearance\\
Filed: 20 March 2084\\
By: Cdr. Paul Navarre, R\and D: Engineering Division, PHALANX, Atlantic Operations Command\\
\paragraph*{Overview}
From our experiences in Mumbai and other stricken cities, we've concluded that the aliens seem to concentrate their efforts on population centres, especially dense urban areas. A majority of engagements have taken place at knife-fighting range. For the purposes of the Excalibur Program, we've chosen several high-performance weapons for our Close Range Combat package.

The Iranian ADA 22 flamethrower (nicknamed 'The Torch of God' by Alliance troops) is a marvel of modern engineering. Unlike the old flamethrowers of the 20th century, it requires no heavy pressure tanks to be carried on a soldier's back -- the ADA is one unit carried in the hands without any hoses or tubes. It simply slots a relatively small 200mm gas canister into the feed, fires its deadly ammunition, and then ejects the empty canister to make room for a reload. The system is ingenious and surprisingly reliable.

The muzzle of the weapon is a powerful pump that squirts gas into the air, then sets off the gas-air mixture by way of four parallel spark igniters, one main unit and three backups. These igniters each emit as many as ten sparks per second in order to ignite the fuel before it disperses. The backups are very important in preventing an explosion due to the special fuel the ADA uses, a substance called Compound 90.

Compound 90 is a new flamethrower fuel that, when injected into the air, creates a slow thermobaric reaction -- a fuel-air ignition rather than a fuel-air explosion -- that can roast living tissue in seconds. C90 inflicts horrific damage on organic targets; the heat generated exceeds 1700 degrees celsius, enough to melt titanium. However, due to its gaseous nature it leaves no burning residue (like napalm) on the target or in the surrounding area and disperses its heat much more quickly. This makes it significantly safer for urban use than napalm-derivative substances.

The ADA 22's main drawbacks are its short range and its complex internals, which are difficult to repair. However, our experienced technicians should have no problem doing maintenance on this model.

For PHALANX use, we have given this flamethrower the classification CRCL-FL.
\paragraph*{Recommended Doctrine}
The CRC-FL is a close-range weapon; the operator needs to be fast in order to get into weapons range and unleash as much hell as possible. It is also a rather weighty weapon despite its relatively slim design. Strength as well as speed is required to wield it effectively.

Flamethrowers also make great ambush weapons -- just make sure that none of our soldiers are in the line of fire.
\paragraph*{Addenda}
This weapon should not be fired if there are civilians in or near the target zone.
\subsubsection*{HMPL Rocket Launcher}
CLASSIFIED LEVEL YELLOW\\
PHALANX Extraterrestrial Response Unit\\
Technical Document, Delta Clearance\\
Filed: 20 March 2084\\
By: Cdr. Paul Navarre, R\and D: Engineering Division, PHALANX, Atlantic Operations Command\\
\paragraph*{Overview}
The South-African MPMDS (Multi-Purpose Missile Delivery System) has a classic, almost surgically clean name that completely belies its purpose. It is the heaviest infantry missile launcher on the market, able to fire anti-tank shells as easily as high-explosive rockets that turn infantry into hamburger. Originally intended as light field artillery, this weapon is fairly new; its only combat experience has been in Mumbai in the hands of Commonwealth troops. It was one of the handful of weapons that eventually turned the tide in the fighting -- and there is a good reason why.

Alien UFOs on the ground emit fantastic amounts of jamming and other EW (Electronic Warfare) activity. In fact, they emit so much of it that ordinary 'smart' missiles are rendered completely ineffective. If one type of alien EW doesn't fool the missile's relatively stupid electronic brain, another will. No amount of tinkering by Earth's military engineers has been able to fix the situation. The MPMDS, however, will remain effective against the alien invaders -- one of the few human missile launchers that can -- because its rockets carry no onboard guidance.

The rockets are 120mm monsters the size of artillery shells. They are fired out of a smoothbore barrel, fin-stabilised in flight, and have a maximum effective range of 70 metres. Surprisingly they are made up of only three parts: a rocket booster, a warhead and an impact trigger. This simplicity allows them to remain effective and reliable in the most hazardous situations. Though the rockets are not very accurate, they will devastate anything in their path, even aliens. It takes only one good hit from an MPMDS rocket to send the enemy flying.

Standard ammunition for this launcher includes: HE (High-Explosive) rockets, AA (Anti-Armour) rockets, and IC (Incendiary) rockets. New types of MPMDS ammo were being researched by Commonwealth manufacturers before the start of the war. We have all their data on file and could use it to create revolutionary new rocket types.

Along with the HPGL Grenade Launcher, this weapon will provide our troops the artillery support they need to survive and win through.

For PHALANX use, we have given this rocket launcher the classification HPML.
\paragraph*{Recommended Doctrine}
The HPML is best used in a support role, providing covering fire for our assault troops with high-explosive and/or incendiary rockets. However, care should be taken that friendly fire incidents do not occur, as this could have disastrous effects in a combat mission.

Friendly troops should be kept away from the rear of the launcher during firing; the hot exhaust gases are extremely dangerous to nearby humans. The HPML should not be fired at targets closer than 8 metres from the shooter. Violating these safety guidelines could result in serious injury or death to the shooter and other members of the team. If a target is closer than 8 metres, any shooter should immediately resort to his sidearm or a combat knife. Anything else would be suicide.
\paragraph*{Addenda}
Along with sniper rifles and grenades, this is one of the few standard-issue human weapons that are fully effective against robotic aliens.
\subsection{Secondary Weapons}
\subsubsection*{Combat Knife}
CLASSIFIED LEVEL YELLOW\\
PHALANX Extraterrestrial Response Unit\\
Technical Document, Delta Clearance\\
Filed: 20 March 2084\\
By: Cdr. Paul Navarre, R\and D: Engineering Division, PHALANX, Atlantic Operations Command\\
\paragraph*{Overview}
The combat knife is a solid, twenty-six centimetre bar of sharp steel that can be used for a number of everyday purposes. As a weapon, the combat knife's edge is reinforced with hard ceramics to give it extra armour-piercing power, and it is balanced for throwing should the need arise. It also has a bayonet ring for affixing the knife to the barrel of an old-style battle rifle.

The inclusion of the combat knife in our arsenal was a subject of hot debate. Mumbai has taught us that even in the most favourable conditions, a knife is no match for the aliens' weaponry. If a knife-equipped soldier isn't shot down before he even reaches his target, he will almost certainly be gutted by one of the wicked alien blades that slice right through armour. Only in a handful of occasions have experienced knife-fighters been able to win out against alien opponents.

Still, the combat knife has saved many a life over the years when firearms became inoperable or ran out of ammo. It has never been rendered obsolete by centuries of progressing technology; even the aliens use bladed weapons. When the chips are down, a knife in hand is still vastly better than a man's own fists.

If a soldier makes it all the way to melee range, this weapon will serve him well.
\paragraph*{Recommended Doctrine}
In combat, the knife is a weapon of last resort. If all other weapons are exhausted and the enemy is at the gates, then a good knife can save the day. However, even in such desperate situations it is risky business; using a knife on an enemy that hasn't been previously wounded or weakened is a fast way to end up in the morgue.
\paragraph*{Addenda}
None.
\subsubsection*{P-12 / 9mm Pistol}
Technical Specifications: P-12 Pistol\\
CLASSIFIED LEVEL YELLOW\\
PHALANX Extraterrestrial Response Unit\\
Technical Document, Delta Clearance\\
Filed: 20 March 2084\\
By: Cdr. Paul Navarre, R\and D: Engineering Division, PHALANX, Atlantic Operations Command\\
\paragraph*{Overview}
With the return of armour to the battlefield, starting with steel helmets in World War 1 and fragmentation vests in Vietnam, pistols have had a harder and harder time keeping up. Due to their far lower muzzle velocity compared to longer-barreled and/or fully automatic weapons, they've had increasing trouble penetrating the new, ever-higher standards of human armour -- much less advanced alien composites. The very concept of the pistol in military use came under fire at one point in the 21st century, and was saved only by the advent of super-high-velocity powder.

Chief among the new generation of super-pistols is the Dolvich DV762 from Russia. It follows the design philosophy of its home country; rugged, reliable power without frills. Its design is extremely basic, and though the materials used in its construction are far stronger to cope with the new powder, the DV762's internals are no more advanced than any pistol of the late 20th century.

The DV762 does not compromise. It isn't a multi-function firearm. It's designed for only one thing: to punch through armour and kill the person inside. In order to do this, the DV fires the ancient 7.62mm Tokarev pistol round, either on semi-automatic or three-round burst mode, from a 12-round detachable box magazine. The Tokarev round is known for its excellent penetration, and it has been significantly upgraded on its return to military service. This pistol can shoot clean through the side of a modern ballistic helmet at ranges of up to 10 metres -- and then out the other side.

Unfortunately, in order to achieve penetration, the DV sacrifices stopping power. The 7.62mm round makes a very clean hole in the enemy, which is the problem; it's very reluctant to fragment or tumble, requiring a direct hit on a vital organ or major artery to incapacitate or kill an enemy.

For the purposes of the Excalibur program, we concluded that a guaranteed minor hit is better than one that may simply bounce off an alien's armour, especially as we have yet to gain a clear picture of how nasty alien armour can get. We need sidearms that we know will be effective in the crunch, and the DV762 is the best of them.

For PHALANX use, we have given this pistol the classification P-12.
\paragraph*{Recommended Doctrine}
The P-12 is primarily a backup weapon. It is a significant step up from the combat knife as a weapon of last resort, and it lets a soldier respond to new close-range threats if the primary weapon is rendered ineffective at such ranges or has run out of ammo.

Ambidextrous soldiers may even consider using two pistols at the same time, though this will negatively impact accuracy and reduce the soldier's already minimal effective range.

A single P-12 should rarely be considered as a primary weapon, as it is outclassed in this role by nearly every other weapon in our arsenal. Its advantages are the advantages of a sidearm -- small size and weight. Still, it may find a use as a primary weapon with field medics and technicians who do not have room for larger weapons.
\paragraph*{Addenda}
Despite good penetration against organics, this weapon performs very poorly against robotic targets.
\subsubsection*{CRC-8 SMG}
Technical Specifications: CRC-8 SMG\\
CLASSIFIED LEVEL YELLOW\\
PHALANX Extraterrestrial Response Unit\\
Technical Document, Delta Clearance\\
Filed: 20 March 2084\\
By: Cdr. Paul Navarre, R\and D: Engineering Division, PHALANX, Atlantic Operations Command\\
\paragraph*{Overview}
From our experiences in Mumbai and other stricken cities, we've concluded that the aliens seem to concentrate their efforts on population centres, especially dense urban areas. A majority of engagements have taken place at knife-fighting range. For the purposes of the Excalibur Program, we've chosen several high-performance weapons for our Close Range Combat package.

Designed and manufactured in mainland China, the Ohm 55 SMG is one of the most frightening weapons to come out of the Second Cold War. It was first prototyped in 2035 by scientists working for the Communist Chinese government. Production models were only trickling into government units by the end of the war, but the rebel and Commonwealth troops quickly learned to respect the Ohm's ferocity.

Its rate of fire at full auto exceeds 1200 rounds per minute. It can chew through a 50-round magazine in three seconds. It fires an upgraded version of the Belgian 5.7mm armour-piercing round, a steel penetrator with aluminium core, which can tear kevlar like paper and tumbles brutally through flesh and bone. Even modern ballistic fibre cannot stop this round at anything closer than 12 metres. The Ohm 55 has dominated the field of SMGs for the past 50 years and will continue to do so for at least the next decade.

The Ohm is highly manoeuvrable with a short barrel and sleek lines, but can suffer from excessive muzzle climb on full auto due to the sheer weight of lead the weapon puts out. Autofire also tends to empty the magazine before the shooter even realises he's holding down the trigger. Still, after nearly 50 years in service around the world, this remains the Ohm 55's only known design flaw.

For PHALANX use, we have given this submachine gun the classification CRC-8.
\paragraph*{Recommended Doctrine}
The CRC-8 is intended for point-blank urban firefights. It will perform very well in this role, but don't expect it to hit the broad side of a barn out to medium range. It can also function as a high-powered sidearm, but may be too bulky for most soldiers to use in this manner.

While the CRC-8 does suffer excessive muzzle climb on full auto, throwing off the aim of even experienced users, it is much more docile in its standard burst mode. Full auto should rarely be considered outside of panic situations.
\paragraph*{Addenda}
Despite good penetration against organics, this weapon performs very poorly against robotic targets.
\subsubsection*{Riot Shootgun}
\paragraph*{Official Description}
\paragraph*{Battle Implications}
The SG-260 riot shotgun has tremendous stopping power at close range and, accordingly, considerable recoil. Its main bulk hardly fits in the holster and the double barrel, though short, slightly protrudes along the soldier leg. To master this secondary weapon, the soldier needs both hands and considerable skill with heavy weapons.
\subsection{Misc}
\subsubsection*{Frag Granade}
CLASSIFIED LEVEL YELLOW\\
PHALANX Extraterrestrial Response Unit\\
Technical Document, Delta Clearance\\
Filed: 20 March 2084\\
By: Cdr. Paul Navarre, R\and D: Engineering Division, PHALANX, Atlantic Operations Command\\
\paragraph*{Overview}
The concept of the fragmentation grenade has changed little since it was first conceived, when Chinese soldiers packed gunpowder into ceramic or metal containers. It only solidified further after the end of the 20th century. Improvements in explosive technology and casing materials have made them a little faster and a little deadlier, but the mechanics are the same: An explosive substance at the heart of the grenade causes the casing (and possibly additional payload such as a layer of wire or white phosphorus coiled around the explosive) to fragment into shrapnel and fly in all directions at high speed, killing or wounding targets in the area of effect. The delayed fuse is ignited by first pulling the pin and then releasing the handle, usually released as the grenade is thrown. It will detonate after a period of several seconds.

For PHALANX purposes, we?ve selected the Australian HG15 as the best of the lot. This grenade is loaded with an extremely conventional charge of C8 solid chemical explosive, an inner layer of coiled wire, and an outer layer of thin plated steel. It's designed to maximise coverage by putting out more shrapnel than any other grenade on the market. The HG15's outer layer is smooth and unbroken, unlike the old 'pineapple' grenades of World War 2, to make it easier to roll across various surfaces. The C8 explosive at its core may be old in design, but it's both highly powerful and can be counted on to work right under harsh conditions. Reliability is an especially important attribute in any kind of hand-held bomb.
\paragraph*{Recommended Doctrine}
Though quite effective against enemy infantry, frag grenades are weapons of opportunity, not a replacement for firearms or portable artillery. Their range is highly limited and their casualty radius is small. However, if the situation calls for close-range indirect fire or something deadly thrown around a corner, the frag grenade is just the thing.

Care should be taken that no friendlies (especially civilians) are caught in the area of effect. In such situations the use of lethal grenades is strongly discouraged; flashbangs should be given preference.

It's recommended that all soldiers carrying non-heavy weapons should be equipped with at least one grenade of some variety -- be it a frag grenade, flashbang, incendiary grenade or other -- in case the need should arise. Two or more are strongly recommended.
\paragraph*{Addenda}
None.
\subsection{Amor}
\subsubsection*{Combat Armour}
CLASSIFIED LEVEL YELLOW\\
PHALANX Extraterrestrial Response Unit\\
Technical Document, Delta Clearance
Filed: 20 March 2084\\
By: Cdr. Paul Navarre, R\and D: Engineering Division, PHALANX, Atlantic Operations Command\\
\paragraph*{Overview}
The use of armour on the battlefield never quite died out completely, though it was rendered ineffective in most forms between the early 18th century and the late 20th. It began to find its way back to common use in World War 1 in the form of steel helmets. This practice continued through WW2, and was later superceded by the invention of kevlar. Today, however, kevlar is thoroughly obsolete in nearly all its forms; it's now used only by civilians and police forces with budget problems. Even more advanced types of 20th-century body armour have been rendered useless by modern weapons. New materials were needed, materials to make armour stronger and its wearers tougher than ever before.

Surprisingly, several ancient files we've unearthed seem to confirm that PHALANX was responsible for some amazing technological breakthroughs in the past, technologies that were later adopted across the world. Every attempt at producing artificial spider silk had failed, but researchers at the PHALANX Pacific Operations Command base finally managed it in 2017. Their technique is still in use today, centred around a device called the 'organic loom'; a large feeding armature supporting hundreds of individual silk glands and spinnerets, designed solely for the mass-production of spider silk.

The first widespread use of military combat armour made from spider silk came as a joint effort by NATO in 2023, after ballistic tests proved that Chinese rounds tore right through their aging standard-issue kevlar vests. The armour itself is a layercake of spider silk and treated ultra high molecular weight polyethylene, giving it astonishing strength and flexibility. The resulting fabric is about 18 times stronger than steel and provides a performance increase over kevlar that is estimated between 300 and 400

The Combat Armour's only disadvantage is its relatively high weight compared to older suits, but this is mainly due to the number of layers required to properly protect against modern weapons. The weight is evenly distributed, making it quite comfortable to wear and much less bulky than experimental nanocomposite armours. This armour will save lives while preserving the soldier's all-important mobility.
\paragraph*{Recommended Doctrine}
Where possible, PHALANX troops should always wear armour whenever they are sent into a combat situation. The Combat Armour should be considered the basic protection no soldier can afford to go without. For some soldiers, the Combat Armour will remain a viable choice compared to heavier, more advanced armours due to the freedom of movement it provides. Snipers and anyone else not expected to be at the front line will be able to make good use of the extra mobility.\textbf{armor types don't influence TUs right now}
\paragraph*{Addenda}
None.
