\section{Options}
The options menu can always be accessed by pressing \keybinding{ESC} till you reach the main screen $\rightarrow$ options.

\subsection{Video}
This section offers you various ways to make UFO:AI look the best way possible to the engine and your system. Please be aware that while most options here can cause improved graphics they can also cause remarkable slow downs your computer.
\subsubsection*{Resolution}
You may choose resolutions between 320x240 and 2048x1536. It might be worth the note that ``after'' some rather rare resolutions like 1280x854 and the like follow which might be interesting for laptop users.\\
You can also set custom resolutions if you set the cvar \cvar{vid\_mode} to \cvarvalue{-1} and use the cvars \cvar{vid\_width} and \cvar{vid\_height} to define your wanted resolution.
\subsubsection*{Fullscreen}
Well, here you either turn fullscreen mode on or off.
\subsubsection*{Texture compression}
\subsubsection*{Texture resolution cap}
\subsubsection*{Show FPS}
If you choose to turn on this option UFO:AI will display current frames per second in the very upper right corner.
\subsubsection*{Texture anisotropy level}
\subsubsection*{Texture Lod}
\subsubsection*{Image filter}
\subsubsection*{Gamma}
Here you may adjust UFO:AIs Gamma factor to your graphic card or monitor settings.

\subsection{Sound}
\subsubsection*{Effects}
Use this fader to adjust effects volume to your neighbours ears.
\subsubsection*{Music}
Use this fader to adjust music volume once you got bored of your private music collection.
\subsubsection*{Mixing rate}
I am not really familiar with sound engineering, so i guess it is ``the more the merrier''\ldots

\subsection{Game}
Besides having the chance to change your ``playername'' the game options also offer more practical opportunities.
\subsubsection*{Start with employees}
Choosing this option will make you start with a set of employees as well as some basic equipment for your soldiers. If you prefer to do really everything on your own, switch to ``no'' here.
\subsubsection*{Start with buildings}
If you say ``yes'' here UFO:AI will equip your first base with standard set of facilities that will do the trick quite well. Perfectionists may choose ``no'' here.
\subsubsection*{Confirm actions}
In order to prevent to fast clicking mistakes or making it easier to play UFO:AI while being drunk you may turn on this option. Doing so will make battlescape showing you the path your soldiers will choose once ordered to move to a certain spot. In order to finally make the soldier in question move there you need to press \keybinding{Enter}.
\subsubsection*{HUD design}
As said at the introduction of battlescape there are two user interfaces available for tactical combats. here you can switch between HUD and altHUD. Please be aware that it is not possible to change the HUD while being in combat (you may change the option, but it won't take affect within the running mission).
\subsubsection*{Center view}
Depending on your setting here the HUD will focus on the selected soldier if you use the team-overview or buttons \keybinding{1} to \keybinding{8} to switch between different soldiers or stay focused on your point of view while switching.
\subsubsection*{Cursor tooltips}
Turn on/off cursor tooltips, indicating the function of various UI elements.
\subsubsection*{Camera scroll}
Adjust camera scroll speed.
\subsubsection*{Camera rotation}
Adjust camera rotation speed.
