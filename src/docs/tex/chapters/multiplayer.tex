\section{Multiplayer}
\subsection{Client}
You can play 'UFO: Alien Invasion' with your friends via LAN or Internet connection. One player controls a human team while another player takes command of the aliens for a bloody head-to-head battle. UFO:AI also supports cooperative team play with multiple players on both sides.
\subsection{Server}
\subsubsection{Listen Server}
You can start a server for you and your friends from within the game menu, just enter the "Multiplayer" menu, load you team like you would do as a client and enter the "Create" menu. Here you can select the map and the gametype. There are also other settings that can have an influence on gameplay - see their tooltips for more information.

\subsubsection{Dedicated Server}
Besides the ability to start a server as listen server, you can also start a server as dedicated server. This is a console only version of the server and only has a text input interface to send commands. Useful commands are \consolecommand{gametypelist}, \consolecommand{maplist} and \consolecommand{map}. You can set the gametype by modifying the \cvar{gametype} cvar.

\subsection{Remote Console}
You can also use the rcon method to change the map - all the server administrator has to do is to set the cvar \cvar{rcon\_password} - the client has to set this cvar to the same value and use the \consolecommand{rcon} as prefix for the normal console commands.

\subsection{Mapcycle}
You can also set up your mapcycle - a map is automatically changed (or restarted when there is no mapcycle) when one team had won the game. The mapcycle is defined in a file called \gamepath{mapcycle.txt} which is in your \gamepath{base/} folder or in \gamepath{~./ufoai/"version"/base}. This file is in the form "map gametype" and each entry is seperated by a newline.\\
There are commands to modify the mapcycle from within the game:
\begin{description}
\item[mapcycleadd] Add new maps to the mapcycle
\item[mapcycleclear] Delete the current mapcycle
\item[mapcyclenext] Start the next map from the cycle
\item[mapcyclelist] Print the current mapcycle
\end{description}
