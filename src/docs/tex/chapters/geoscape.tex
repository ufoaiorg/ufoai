%-*-texinfo-*-
%This is part of the "UFO:Alien Invasion"-Reference Manuals Tex sources.
%Copyright (C) 2006 Eric Goller.
%See the file ufo-manual_EN.tex for copying conditions.

\chapter{Geospace}

\section{Worldmap - an overview} 
Welcome to Geoscape! As said before UFO:AI distincts between two major aspects of the game - macromanagement and tactical combat. To put it simple one could say: Combat is where you earn the bucks (besides honor of cause ... ), Geoscape is where you spend em.

Geoscape itself basicly consist of two screens. The first one is the world map. This is where you get right after starting a new campain and is used to get the bigger picture as well as coordinating combat missions and intercepting enemy UFOs. Second is the base overview, where you improve infrastructures and order important decisions about equipment, research and production. In the following we will take a closer look at both of them.

Taking the following screenshot we will have a look at what we might want to use the world map for.
Pleate take notice of the fact that it is divided into day and night zones which actualy influence any combat mission you get into (this boarderline changes its shape according to the seasons as the relation earth-sun changes).

\subsection{Buttons}
\paragraph*{Status screen}
Here some general information(e.g. stats, discriptions) show up depending on the context. What is shown in detail will be explaint in the following.
\paragraph*{Statistics / Ufopedia / New base}
If you hover over those registers three different buttons will show up. While the very left one leads to some more detailed statistics about your attemp to save the world (such as mission overview, happiness of the countries paying your efforts) the middle one will open up Ufopedia. Ufopedia is a comprehensive collection of usefull information about items, technologies, damage types and others. As your research proceeds Ufopedia grows as well, so make sure you check the latest news on your enemy's every now and then. Finally the right button gives you the chance to establish a new base anywhere on the map. A new base, once you installed the required structures, gives additional radar-range, research and production capacities as well as new hangars for your aircrafts and iscompletely equal (also in administration) to your first base.
\paragraph*{Date}
Gives you the current date, so you know when its close to pay day. Please keep in mind that for mankind time is kind of running up. While you, in principle, have unlimited time at your hand, in fact aliens get stronger and better equipped as the game proceeds and you will have to catch up with them in order to beat them and save your beloved homeworld.
\textbf{aliennation ???}
\paragraph*{time}
Well, as you location isn't nailed down to one certain base this is just to illustrate how fast time proceeds. See also next paragraph
\paragraph*{Gamespeed switch}
This is where you can adjust the gamespeed from 5secs (which is in fact pausing the game) over 5mins up 1day steps. Whatever you put here, while you are in combat time is stoped and it will be all the same when you return from the field of honor.
\paragraph*{Credits}
Should be quite self-explaining. Never forget, you can't spend what you don't have.
\paragraph*{Options}
Gets you to the Options-menue where you can load and save your game as well as start a new one.
Through "exit" you reach the main menue where you can change game settings and continue your current game (via singleplayer \hookrightarrow continue)
\paragraph*{news and extended news}
The permanent news line in the upper left always represents the latest newline (such as promotions / cashflow / attacks / ufo-sightings) while the extended news button pops up a list of the last 20 newlines.
So whenever you notice news, make sure to check the button as well so you don't miss anything. 
\paragraph*{Bases}
Those yellow houses represent your bases. Circles around them (poping up later in the game) represent their radars range.  If you want to "enter" a base just click on its symbol.
\paragraph*{Your dropship}
This is the one that gets you squad to action. Clicking on it once brings up some general data about it (like fuel, speed, status and amount asigned soldiers) to the status screen. A second click while it's selected opens a submenue where you may give/change orders,e.g. sending it back home.
\paragraph*{Your intercepors}
Those fast ships job is to take enemy UFOs down. If it catches up with one the dogfight is going to be calculated based on both ships equipment and the result shown on screen. Just like one paragraph before a single click selects the interceptor (a further click to a certain spot on the map will order it to move it there) printing some general information in the status window. A second click while it is selected brings up a window where you might give more advanced orders to your ship.
\paragraph*{Upcomig missions}
This is where the action waits. Selecting a mission will give you a short discription on the status screen while a second one makes you select a ship to bring in the troops you want.


\section{Your base}
\subsection{Buttons}
%wahrscheinlich umfangreichster teil
\subsection{What to do now ... ?}

\section{Gamemechanics / Managment}
%evtl. stark ausbauen
\subsection{Research}
\subsection{Promotions}