\section{Your base}
Your bases have to fulfil a wide range of tasks, ranging from researching and producing new equipment, gathering background information on the invaders and supplying the infrastructure to react to any alien incursion. You can change the name of your bases by clicking on the pen-icon right next to its name shown on base screen. You can also cycle through all your current bases using the arrow icons. In the following we will list all relevant screens so you can get familiar with the base management system.

\subsection{Buildings}
This is where you order the construction of additional facilities for your base.  Building laboratories will allow you to increase your research cap, hospitals will allow you to hasten your soldiers' healing.
Before you place a new building, make sure you have read its ufopedia entry. There you can find out if the new building requires additional buildings to be constructed, or what it's actually used for. Another important aspect when expanding your base is building time -- buildings vary quite a bit in this regard. Keep in mind that at least a power plant and command centre are needed for most other buildings to be useful. Don't forget, new bases can be built in other locations, so you don't need to place all facilities in one site. You will want to consider this, as space on individual bases is limited.

\subsection{Aircraft}
This menu brings up a screen where you manage the aircraft at that particular base. This includes not only equipping your vessels with the latest available equipment, but also buying new ones or transferring them to another base. You can circle through all your aircraft using the left and right arrow icons in the window displaying the current aircraft. You can also call a ship back to base or launch it from here, although you're more likely to want to do this from the Geoscape display.

Probably the most important sub-menu here is ``Equip Aircraft''.  This brings up a screen which allows you to choose which soldiers to assign to your selected aircraft. Obviously this is quite important when it comes to your dropship. A standard dropship has room for 8 soldiers, and you will almost always want to use all of them. In order to choose the best soldiers for an upcoming mission you are provided with an picture of your selected character and his / her statistics. A simple click on the `X' or$\surd$ assigns or removes the selected soldier from the current ship. You may also rename your fighters using the ``edit'' button in the upper right, just next to current soldier's name. Also please notice that while you can assign one soldier to an interceptor ship, this is unlikely to do you any good.

Once you have made your decision on who to take to the battlefield, you must confirm your selection using the button in the very bottom right corner.  At this point an inventory screen will come up. You can re-do your troop selection as often as you want, provided the ship in question hasn't left the base.

At the inventory screen, you can equip your soldiers for their upcoming missions. The different sections of this screen should be quite self explanatory; nevertheless we will comment on some of its basic features. In the upper left you see all soldiers assigned to the current aircraft. On the opposite side of the screen, you see the soldier with his / her inventory. The amount of space an item requires is represented by the number of ``squares'' covered. The biggest part of the screen is used by your base's item stock. In order to make it easier to use the rather big amount of items you can choose one of 4 categories (primary/secondary/misc/armor) to be displayed here.  Simple drag \and drop gets any item from bases stock to the specific inventory of your soldiers. Weapons shown with a red background lack the required ammo and aren't useable. You may equip them anyway but unless you get the required ammunition from somewhere else they won't be of any use. In order to assist you in your task to equip every soldier with a weapon he can handle effectively the lower left shows the soldiers statistics (for details on stats please refer to the appendix or ufopedia). Please keep in mind that some weapons utilise two weapon proficiencies depending on the chosen firemode. Alternatively to the soldier's stats window you can change this to an object details view which presents the basic stats (one / two handed, round per clip, firemodes, damage, etc.) of an item. For details on damage and firemodes of a weapon you need to view the details of the according clip / ammunition, as some weapons can be equipped with different types of ammo. A simple click on the arrow symbol in the very bottom right corner confirmes your selections and gets you back to the aircraft screen.

\subsection{Buy / Sell Equipment}
Here you can get new equipment from the global market or get rid of any item you don't have further use for. Please be aware that the items not carried by your soldiers at the end of a mission are sold automatically. Details will be displayed on missions summery screen. If you want to use the items captured you can simply buy them back here. As there is no differences between purchasing and selling prices you won't lose money doing so. This is very likely to be changed once the whole economy thing is set up right till then global market can be exploited as a kind of unlimited equipment storage. Please notice that the amount of any kind of items available may change in the course of the game as your reputation in the world changes. In order to help keeping an overview all items are sorted into four categories again (primary, secondary, misc, armor).

\subsection{Transfer}
Here you can transfer your equipment between different bases.
%moretocome

\subsection{Research}
As research is a critical factor in your attempts to defend earth against the alien threat it is essential to keep your R \and D department busy not only in order to get the lastest weapon technology but to gather background information about your enemy and ways to finally defeat him. The basic features of the research screen are rather simple. While the left part gives all possible research options the right part shows details on the selected subject. In order to discover new research options its usually necessary to capture either at least one kind of the regarding item or a certain key item that offers new information about the alien threat. Sometimes a simple prototype of some alien tech is not enough to get your research started. In such cases the research option is given in grey letters as it requires further research on some other more basic field beforehand. The concrete dependencies for each technology are given in its details shown on the right side of the screen.

To assign a given amount of scientists to a research project just use the left / right arrow-icons next to the technology in question. The left arrow will add scientist to the research while the right one will decrease the amount of scientists working on that project.\\
The actual progress-status is given in the left window. Hint: while it is possible to work on several technologies at the same time in most cases its a better strategy to focus on one research at a time.

\subsection{Production}
Here you can built equipment that is not available on global market or a result of your research departments efforts. To order an item to be build simply select it on the left part of the screen and adjust the amount to be build using the arrow-icons under its image on the right part. Also, please notice that the production cost is taken from your cash when one item is started. For example while: 3 assault rifles cost 63000 you need only 21000 to start production.

\subsection{Hire employees}
Using this screen you can add further personal to your organisation. While especially in the beginning people do not trust in your ability to counter the aliens they might be more enthusiastic (and therefore willing to work for you) as you proceed in the game. On the left side you find all members of one group (soldiers/medics/workers/scientist) listed. clicking on the `X' or $\surd$ hires or fires them. You can discard / select them as often as you want, they will never get angry at you. But please be aware that personnel you hire in one base won't be accessible from another base. So if you want to fire someone make sure you are in the corresponding base. Also you should keep in mind that the amount of personnel that can work in your base might be limited by the base's housing or working facilities. AFAIK this is not the case right now. also it doesn't seem possible to hire more than 19 persons of one group for the simple reason that there is no way to scroll down the list.

